\chapter{使用说明}


\section{文件组织结构}
\begin{description}
\item[main.tex] \textbf{ 主文档};
\item[mainref.bib] BibTeX格式的参考文献数据;
\item[bstutf8.bst] 参考文献样式,不需要修改;
\item[content] 存放摘要、各章节及谢辞等文档的目录;
\item[figures] 存放论文中插入的图片的目录。
\end{description}
\section{使用前准备}
在使用本模板编译\LaTeX 文档之前,需安装以下软件:
\begin{itemize}

\item{\bf texlive 2020} 本模板基于texlive 2020包含的CTeX宏包编写,无法保证能在更早期的版本上编译。可以从\url{http://tug.org/texlive/}下载安装。
\end{itemize}

还需要掌握\LaTeX 基础知识。可阅读《一份不太简短的\LaTeX 介绍》(\url{http://www.latexstudio.net/archives/6058}),或购买刘海洋编著的《\LaTeX 入门》,或胡伟的《\LaTeXe 完全学习手册》。

\section{编译说明}
需要对主文档执行四次编译,通过 xelatex + bibtex + xelatex + xelatex 生成带有完整目录和参考文献信息的 PDF 文件。

\section{查重须知}
知网查重仅需要正文和参考文献,可注释掉无关的包含文件代码后编译。必要时可使用pandoc(\url{http://www.pandoc.org/})将\LaTeX 文档 转换为word文档以供查重之用。

\section{后续更新}
由于水平有限,目前本模板仅为一个demo,尚有多项工作未完成,希望有志同道合的校友共同完成此模板。

\section{已知问题}
\begin{itemize}
	\item[1)] \sout{目录页:目录较少时,多出空白页,同级目录较多时,存在对齐问题。}
	\item[2)] \sout{跨页对齐存在问题。}
	\item[3)] \sout{攻读学位期间学术成果未完成格式设计,需要手动设置格式。}\quad 手动设置格式即可。
	\item[4)] 封面间距设计随意,大多为了美观。
	\item[5)] 学校一共给出了6类硕士毕业论文规范,本模板仅有一类(实际上差异很小,略加改动即可)。
	\item[6)] \sout{字体大小警告。}\quad 剩下附录代码注释的字体加粗失败警告,待处理。
\end{itemize}
\section{项目地址}
\url{https://github.com/9527567/QDULaTeXthesis}